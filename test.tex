% Options for packages loaded elsewhere
\PassOptionsToPackage{unicode}{hyperref}
\PassOptionsToPackage{hyphens}{url}
%
\documentclass[
  11pt,
]{article}
\usepackage{amsmath,amssymb}
\usepackage{lmodern}
\usepackage{iftex}
\ifPDFTeX
  \usepackage[T1]{fontenc}
  \usepackage[utf8]{inputenc}
  \usepackage{textcomp} % provide euro and other symbols
\else % if luatex or xetex
  \usepackage{unicode-math}
  \defaultfontfeatures{Scale=MatchLowercase}
  \defaultfontfeatures[\rmfamily]{Ligatures=TeX,Scale=1}
\fi
% Use upquote if available, for straight quotes in verbatim environments
\IfFileExists{upquote.sty}{\usepackage{upquote}}{}
\IfFileExists{microtype.sty}{% use microtype if available
  \usepackage[]{microtype}
  \UseMicrotypeSet[protrusion]{basicmath} % disable protrusion for tt fonts
}{}
\makeatletter
\@ifundefined{KOMAClassName}{% if non-KOMA class
  \IfFileExists{parskip.sty}{%
    \usepackage{parskip}
  }{% else
    \setlength{\parindent}{0pt}
    \setlength{\parskip}{6pt plus 2pt minus 1pt}}
}{% if KOMA class
  \KOMAoptions{parskip=half}}
\makeatother
\usepackage{xcolor}
\usepackage[margin=1in]{geometry}
\usepackage{graphicx}
\makeatletter
\def\maxwidth{\ifdim\Gin@nat@width>\linewidth\linewidth\else\Gin@nat@width\fi}
\def\maxheight{\ifdim\Gin@nat@height>\textheight\textheight\else\Gin@nat@height\fi}
\makeatother
% Scale images if necessary, so that they will not overflow the page
% margins by default, and it is still possible to overwrite the defaults
% using explicit options in \includegraphics[width, height, ...]{}
\setkeys{Gin}{width=\maxwidth,height=\maxheight,keepaspectratio}
% Set default figure placement to htbp
\makeatletter
\def\fps@figure{htbp}
\makeatother
\setlength{\emergencystretch}{3em} % prevent overfull lines
\providecommand{\tightlist}{%
  \setlength{\itemsep}{0pt}\setlength{\parskip}{0pt}}
\setcounter{secnumdepth}{-\maxdimen} % remove section numbering
\newlength{\cslhangindent}
\setlength{\cslhangindent}{1.5em}
\newlength{\csllabelwidth}
\setlength{\csllabelwidth}{3em}
\newlength{\cslentryspacingunit} % times entry-spacing
\setlength{\cslentryspacingunit}{\parskip}
\newenvironment{CSLReferences}[2] % #1 hanging-ident, #2 entry spacing
 {% don't indent paragraphs
  \setlength{\parindent}{0pt}
  % turn on hanging indent if param 1 is 1
  \ifodd #1
  \let\oldpar\par
  \def\par{\hangindent=\cslhangindent\oldpar}
  \fi
  % set entry spacing
  \setlength{\parskip}{#2\cslentryspacingunit}
 }%
 {}
\usepackage{calc}
\newcommand{\CSLBlock}[1]{#1\hfill\break}
\newcommand{\CSLLeftMargin}[1]{\parbox[t]{\csllabelwidth}{#1}}
\newcommand{\CSLRightInline}[1]{\parbox[t]{\linewidth - \csllabelwidth}{#1}\break}
\newcommand{\CSLIndent}[1]{\hspace{\cslhangindent}#1}
\usepackage{setspace}\doublespacing
\usepackage{lineno}\linenumbers
\ifLuaTeX
  \usepackage{selnolig}  % disable illegal ligatures
\fi
\IfFileExists{bookmark.sty}{\usepackage{bookmark}}{\usepackage{hyperref}}
\IfFileExists{xurl.sty}{\usepackage{xurl}}{} % add URL line breaks if available
\urlstyle{same} % disable monospaced font for URLs
\hypersetup{
  pdftitle={Untitled},
  hidelinks,
  pdfcreator={LaTeX via pandoc}}

\title{Untitled}
\author{}
\date{\vspace{-2.5em}}

\begin{document}
\maketitle

Elucidating the selective forces that generate patterns of phenotypic
diversity is a major goal in evolutionary biology. For species that
utilize distinct habitats, disentangling the causes of phenotypic
differentiation is essential for our understanding of how natural
selection operates and how evolution proceeds. In this study, we
evaluated the role of potential drivers of body shape differentiation in
the geckos of the genus \emph{Pristurus}. To this end, we compared
allometric trends and levels of integration among \emph{Pristurus}
occopying distinct habitats, interrogated allometric patterns at both
the static and evolutionary levels, and related these trends to
diversification in body form. Our findings have several important
implications for how ecological specialization, phenotypic integration,
and body form evolution along allometric trajectories relate to patterns
of phenotypic diversity generally, and the evolution of phenotypic
diversification in \emph{Pristurus} in particular. \hfill\break

First, our analyzes revealed that patterns of body shape allometry and
morphological integration are relatively distinct in ground-dwelling
\emph{Pristurus} lizards, as compared with \emph{Pristurus} occupying
other habitats. Specifically, we found that multivariate vectors of
regression coefficients differed significantly from what was expected
under isometry (Table 2) for taxa utilizing all habitat types (ground,
rock, tree), indicating that in \emph{Pristurus}, allometric patterns
predominate. Further, our interrogation of allometric trends revealed
differences between habitat types, where ground-dwelling
\emph{Pristurus} displayed steeper (i.e., positively allometric) trends
for both head and limb traits, while rock and tree-dwelling taxa
displayed shallower (negatively allometric) trends for head traits and
more varied patterns for limb proportions. Biologically, these patterns
revealed that not only does shape differ between large and small
\emph{Pristurus}, but this pattern differs across habitat types.
Specifically, large ground-dwelling \emph{Pristurus} present
disproportionately larger heads and longer limbs relative to large
individuals in other habitats, while small ground-dwelling
\emph{Pristurus} exhibit disproportionately smaller heads and shorter
limbs (Figure 3). These findings are consistent with previous work at
the macroevolutionary level, (Tejero-Cicuéndez et al. 2021), where large
ground species were also found to display disproportionately large heads
and long limbs. \hfill\break

Second, our findings revealed that in rock-dwelling \emph{Pristurus} a
converse pattern was found, where smaller individuals displayed
relatively larger heads, while larger individuals proportionately
smaller heads for their body size. These allometric patterns are also
corresponded with findings at macroevolutioanry scales (Tejero-Cicuéndez
et al. 2021), where similar patterns at the species level were observed.
Additionally, analyses by Tejero-Cicu'\{e\}ndez et al. (2021) indicated
that the rock habitat was the most likely ancestral condition in the
group, with subsequent colonization of ground habitats. If this
hypothesis is correct, it implies a concomitant evolutionary shift in
allometric trajectories (sensu Adams and Nistri 2010) from a more
rock-like pattern to that found in ground-dwelling taxa, which our
analyses reveal (Figure 3). This further suggests that the segregation
in body size and shape through differential allometric relationships
across habitats responds to adaptive dynamics concerning the
colonization of new habitats. Thus, in \emph{Pristurus}, there is
support for the hypothesis that colonization of ground habitats has been
a trigger for morphological change (Tejero-Cicuéndez et al. 2021), as
there appears to be a link between shifts in allometric trajectories as
a result of habitat-induced selection, and differential patterns of body
shape observed across taxa.

\textbf{NOTE: one thing I am not sure about is WHY we see }

Second, \ldots.

Indeed, \ldots.

Additionally, \ldots{}

Finally, \ldots.

\hypertarget{refs}{}
\begin{CSLReferences}{1}{0}
\leavevmode\vadjust pre{\hypertarget{ref-AdamsNistri2010}{}}%
Adams, D. C., and A. Nistri. 2010. Ontogenetic convergence and evolution
of foot morphology in european cave salamanders (family:
plethodontidae). BMC Evolutionary Biology 10:1--10. BioMed Central.

\leavevmode\vadjust pre{\hypertarget{ref-Tejero-Cicuendez2021}{}}%
Tejero-Cicuéndez, H., M. Simó-Riudalbas, I. Menéndez, and S. Carranza.
2021. \href{https://doi.org/10.1098/rspb.2021.1821}{Ecological
specialization, rather than the island effect, explains morphological
diversification in an ancient radiation of geckos}. Proceedings of the
Royal Society B: Biological Sciences 288:20211821.

\end{CSLReferences}

\end{document}
